\noindent The  MMDP model proposed in the previous section results in a very large search space even for small-sized problems. For example, with 8 players and 17
tasks in a 50$\times$55 grid, the number of possible states is more
than $2\times 10^{400}$. Therefore, it is practically impossible to compute the optimal solution. In such cases, it is therefore better to consider approximate solution approaches that result in high quality allocations.  The point of departure for our approximate solution comes from the observations that, when making a decision, the responders first
need to {\em cooperatively} select a task to form a team with
others (i.e., agree on who will do what), and they can {\em independently} compute the best path to
the task. In our planning algorithm, we use this observation to
decompose the decision-making process into a hierarchical structure
with two levels:
\begin{itemize}
  \item At the top level, a task allocation algorithm is run for
      the whole team to assign the best task to each responder
      given the current state of the world.
  \item At the bottom level, given a task, a path planning
      algorithm is run for each responder to find the best path
      to the task from his or her current location.
\end{itemize}

Furthermore, not all states are relevant to the problem (e.g., if a
responder gets injured, he or she is incapable of doing any task in
the future and therefore his or her states are irrelevant to other
responders) and we only need to consider the reachable states given
the current global state $S$ of the problem. Hence, given the current state,
we compute the policy online only for reachable states. This saves
a lot of computation because the size of the reachable states is
usually much smaller than the overall state space (\textbf{Feng: give an example of the reduction}). Another
advantage of online planning is that it allows us to tweak the
model as more information is obtained or unexpected events happen.
For example, if the wind speed increases or the direction of wind increases, the uncertainty about
the radioactive cloud may increase. If a responder becomes tired,
the outcome of his or her actions may be liable to more uncertainty.

The main process of our online hierarchical planning algorithm is
outlined in Algorithm~\ref{alg:coordination}. The following
sections will describe the procedures of each level in more detail.

\begin{algorithm}[t]
  \caption{Team Coordination}\small
  \KwIn{the MMDP model and the current state $s$.}
  \KwOut{the best joint action $\vec{a}$.}
  \tcp{The task planning}
  $\{ t_i \} \gets$ compute the best task for each responder $i\in I$ \;
  \ForEach{$i\in I$} {
    \tcp{The path planning}
    $a_i \gets$ compute the best path to task $t_i$ \;
  }
  \Return{$\vec{a}$}
  \label{alg:coordination}
\end{algorithm}

\subsection{Task Allocation}
\label{sec:taskplanning}

\noindent As described in Section \ref{sec:model}, each responder $p_i$ has a
specific role $r_i \in Roles$ to determine which task he or she can perform and  a task $t$ 
can only be completed by a team of responders with the required
roles $Roles(t)$. If, at some point in the execution of a plan, a responder $p_i$ is incapable of performing a
task (e.g., because she is tired or suffered a high radiation does), he or she will be removed from the set of responders under consideration that is $I = I \setminus p_i$ if $pi$. This information can be obtained from the state $s \in S$.  When a task is completed by a chosen coalition, the task is simply removed from the set, that is $T = T\setminus t_k$ if $t_k$ has been completed. 

Now, in order to capture the efficiency of groupings of responders at performing tasks, we define the notion of a coalition $C$ as a subset of responders, that is, $C \subseteq I$.\footnote{Here coalitions are not considered in the game-theoretic sense as all agents and coalitions aim to maximise the global objective.} Thus, we can identify all possible coalitions $\{ C_{jk} \}$ for each task $t_j$ where $\{r_i | p_i \in C_{jk}\} = Roles(t_j)$. Crucially, we define the value of a coalition $v(C_{jk})$ that reflects the level of performance of a coalition $C_k$ in performing task $t_k$.  Then, the goal of task allocation algorithm is to assign a task to each
coalition that maximises the overall team performance (\textbf{Feng: formalise what you mean by team performance}) given the current
state $s$. In what follows, we first detail the procedure to compute the value of all coalitions that are valid in a given state and then proceed to detail the main algorithm to allocate tasks. Note that these algorithms take into account the uncertainties captured by the MMDP.


\subsubsection{Coalition Value Calculation}
The computation of 
values  $v(C_{jk})$ for each coalition $C_{jk}$ is challenging because not all tasks can be
completed in one shot (\textbf{Feng: what do you mean by one shot??}) and the policy after completing task $t_j$
must be computed as well and this is time-consuming. Here, we propose to estimate the value through several simulations. This is much
cheaper computationally because we do not need to compute the complete policy in order to come up with a good estimate of the value of the coalition.
According to the central limit theorem, as long as the number of
simulations is sufficient large, the estimated value will converge
to the true coalition value. The main process is outlined in
Algorithm~\ref{alg:tp}.
\begin{algorithm}[htbp]\small
  \caption{Coalition Value Calculation}
  \KwIn{the current state $s$,
  a set of unfinished tasks $T$,
  and a set of free responders $I$.}
  \KwOut{a task assignment for all responders.}
  $\{ C_{jk} \} \gets$ compute all possible coalitions of $I$ for
  $T$ \;
  \ForEach{$C_{jk} \in \{C_{jk}\}$}{
    \tcp{The $N$ trial simulations}
    \For{$i=1$ \KwTo $N$}{
        $(r, s') \gets$ simulate the process with the starting \\\Indp state $s$
        until task $k$ is completed by the responders in $C_{jk}$ \; \Indm
        \If{$s'$ is a terminal state} {
            $v_i(C_{jk}) \gets r$ \;
        } \Else {
            $V(s') \gets$ estimate the value of $s'$ with MCTS \;
            $v_i(C_{jk}) \gets r + \gamma V(s')$ \;
        }
    }
    $v(C_{jk}) \gets \frac{1}{N} \sum_{i=1}^{N} v_i(C_{jk})$ \;
  }
  \Return the task assignment computed by Equation~\ref{eq:cf}
  \label{alg:tp}
\end{algorithm}
(\textbf{Feng: please add line numbers to your algorithms and make sure you refer to each line in the algorithm in the text as well.})
In each simulation, we first assign the responders in $C_{jk}$ to
task $t_j$ and run the simulator starting from the current state
$s$. After task $t_j$ is completed, the simulator returns the sum
of the rewards $r$ and the new state $s'$. If all the responders in
$C_{jk}$ are incapable of doing other tasks (e.g., having received too high
a radioactive dose), the simulation is terminated. Otherwise, we
estimate the expected value of $s'$ using Monte-Carlo Tree Search
(MCTS), which provides good tradeoff between exploitation and
exploration of the policy space and has been shown to be efficient
for large MDPs~\cite{?}. The basic idea of MCTS is to maintain a
search tree where each node is associated with a state $s$ and each
branch is a task assignment for all responders. After $N$
simulations, the averaged value is returned as an approximation of
the coalition value.

 As we assume that the role of a responder and the role requirements of each task is static, we can
compute all possible coalition values offline and therefore, in the online phase,
we only need to filter out the coalitions for completed tasks and those containing incapacitated responders to compute the coalition set $\{ C_{jk}
\}$.  

\subsubsection{Coalitional Task Allocation}
Given the coalition values computed above, we then solve the following
optimisation problem to find the best solution:
\begin{equation}
  \begin{array}{lll}
    \max\limits_{x_{jk}} & \sum_{j, k} x_{jk} \cdot v(C_{jk}) & \\[2pt]
    \mbox{s.t.} & x_{jk} \in \{0, 1\} & \\[2pt]
    & \forall j, \sum_{k} x_{jk} \leq 1 & \mbox{(i)} \\[2pt]
    & \forall i, \sum_{j, k} \delta_i(C_{jk}) \leq 1 & \mbox{(ii)}
  \end{array}
  \label{eq:cf}
\end{equation}
where $x_{jk}$ is the boolean variable to indicate whether
coalition $C_{jk}$ is selected for task $t_j$ or not, $v(C_{jk})$
is the characteristic function for coalition $C_{jk}$, and
$\delta_i(C_{jk}) = 1$ if responder $p_i\in C_{jk}$ and 0
otherwise. In the optimisation, Constraint (i) ensures that a task
$j$ (\textbf{Feng: do you mean $t_j$ - be consistent with notation}) is allocated at most to only one coalition (a task does not
need more than one group of responders). Constraint (ii) ensures
that a responder $p_i$ is assigned to only one task (a responder cannot
do more than one task at the same time). This is a standard MILP
that can be efficiently solved  using standard solvers such as IBM ILOG's CPLEX. 

(\textbf{Feng: can you try to integrate the following text in the two subsections above? It's not clear where the UCT equation should go..})
In the task planning level, ``completing a task by a responder'' is
a macro action, assuming that each responder can find the best path
to the task (Section~\ref{sec:pathplanning} gives more detail about
how to compute this). Thus, the main step of implementing MCTS is
to compute an assignment for the free responders (A responder is
free when he or she is capable of doing tasks but not assigned to
any task) at each node of the search tree. This can be computed by
Equation~\ref{eq:cf} using the coalition values estimated by the
UCT heuristic~\cite{?}:
\begin{equation}
  v(C_{jk}) = \overline{v(C_{jk})} + c\sqrt{\frac{2N(s)}{N(s, C_{jk})}}
\end{equation}
where $\overline{v(C_{jk})}$ is the averaged value of coalition
$C_{jk}$ at state $s$ so far, $c$ is a tradeoff constant, $N(s)$ is
the visiting frequency of state $s$, and $N(s, C_{jk})$ is the
frequency that coalition $C_{jk}$ has been selected at state $s$.
Intuitively, if a coalition $C_{jk}$ has bigger averaged value
$\overline{v(C_{jk})}$ or is rarely selected ($N(s, C_{jk})$ is
smaller), it has higher chance to be selected in the next visit of
the tree node.

\subsubsection{Adapting to Responders}
One main advantage of our approach is that
it can easily incorporate the preferences of the
responders. For example, if a responder rejects  a task allocated to it by the planning agent, we
simply filter out the coalitions for the task that contain the
responder. By so doing, the responder will not be assigned to the
task. Moreover, if a responder prefers to do the tasks with another
responder, we can increase the weights of the coalitions that contain
them in Equation~\ref{eq:cf} (By default, all coalitions have
identical weights of 1.0). Thus, our approach is adaptive to
various preferences of human responders. In particular, we show how the adaptive capability of our algorithm is used in AtomicOrchid in a real-world deployment (in Section \ref{sec:evaluation})
Next we show how the path of each responder is computed taking into account real-world uncertainties.
\subsection{Path planning}
\label{sec:pathplanning}
\noindent In the path planning phase, we compute the best path for a
responder to her assigned task. This phase
is stochastic as there are uncertainties in the radioactive cloud
and the responders' actions. We model this problem as a
single-agent MDP that can be defined as a tuple, $\mathcal{M}_i =
\langle S_i, A_i, P_i, R_i \rangle$, where:
\begin{itemize}
  \item $S_i = S_r \times S_{p_i}$ is the state space. In this
      level, responder $p_i$ only need to consider the states
      of the radioactive cloud $S_r$ and his or her own states
      $S_{p_i}$ in the MMDP.
  \item $A_i$ is the set of $p_i$'s actions. In this level,
      responder $p_i$ only need to consider her moving
      actions.
  \item $P_i = P_r \times P_{p_i}$ is the transition function.
      In this level, responder $p_i$ only need to consider the
      spreading of the radioactive cloud $P_r$ and the changes
      of his or her locations and health levels when moving in
      the filed $P_{p_i}$, which are defined earlier in the
      MMDP.
  \item $R_i$ is the reward function. At this level, responder
      $p_i$ only needs to consider the cost of moving to a task and the
      penalty of receiving high radiation doses.
\end{itemize}
This is a typical MDP that can be solved by many state-of-the-art
MDP solvers~\cite{?}. We choose the Real-Time Dynamic
Programming (RTDP)~\cite{?} approach because it particularly fits  our
problem, that is, a goal-directed MDP with large number of states. Instead
of exploring the whole state space, RTDP only visits the states
that are reachable from the initial state $s^0$ (the start location
of the responder). The main process is outline in
Algorithm~\ref{alg:pp}. If the goal is not reached in a number of
iterations, we assume there does not exist a path between the start location
of the responder and the location of the task  (either there are obstacles
on the path or the responder will be killed by the radioactivity on
the road).
\begin{algorithm}[htbp]
  \caption{Path Planning}
  \KwIn{the starting state $s^0$ and the goal state $s^g$.}
  \KwOut{a path from the starting location to the goal.}
  $s \gets s^0$ \;
  \Repeat{$s = s^g$}{
    \ForEach{$a\in A_i$}{
        $Q(s, a) \gets R_i(s, a) + \sum_{s'\in S_i} P_i(s'|s, a)
        V(s')$ \;
    }
    $a \gets \arg\max_{a'\in A_i} Q(s, a')$ \;
    $V(s) \gets Q(s, a)$ \;
    $s' \sim P_i(s'|s, a)$ \;
    $s \gets s'$ \;
  }
  \Return{$Q$}
  \label{alg:pp}
\end{algorithm}
(\textbf{Feng: please add line numbers to this algorithm and explain in the text what is going on.}
There are several techniques we use to speed up the convergency of
RTDP. In our problem, the terrain of the field is static. Thus, we
can initialize the value function $V(s)$ using the cost map (\textbf{Feng: what is the cost map???})
computed offline without considering the radioactive cloud. The
cost map stores the shortest path and the cost value between any
two points in the map. This helps RTDP quickly navigate among
the obstacles (e.g., buildings, water pools, blocked roads) without
getting trapped in dead-ends during the search. Another speed up is also possible if, when traversing the reachable states (i.e., $s'\in S_i$ in
Algorithm~\ref{alg:pp}), we only consider the responder's current
location and the neighbouring points, since $P_i(s'|s,a) = 0$ for
other points. This will further speed up the algorithm where the
main bottleneck is the huge state space.

(\textbf{Feng: just provide a graph depicting the performance of the algorithm compared to greedy and say more extensive evaluations are beyond the scope of this paper as the focus is on the use of the algorithm in a real-world deployment to test how humans take up advice computed in sophisticated ways by an agent-based planner)}.
