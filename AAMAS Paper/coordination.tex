Given a reasonable size of the game, the MMDP model can be huge.
For example, with 8 players and 17 tasks in a 50$\times$55 grid,
the number of possible states is more than $2\times 10^{400}$.
Thus, it is computational intractable to find the optimal plan of
the game. One useful observation is that, when making a decision,
each player first selects a task to form a team and then move to
the task location independently. In our planning algorithm, we use
this hierarchical structure to decompose the decision making
process into two level:
\begin{itemize}
  \item In the higher level, task planning algorithm is run for
      the whole team to assign the best task to each players
      given the current state.
  \item In the lower level, by given a task, path planning
      algorithm is run for each player to find the best path to
      the task from his current location.
\end{itemize}
Given the huge state space, it is intractable to find the plan for
all states offline. Therefore, we only compute plans for the
queried states online. The main process of the online hierarchical
planning algorithm is outlined in Algorithm~\ref{alg:coordination}.
The following subsections will describe the algorithms in each
level in more detail.

\begin{algorithm}[t]
  \caption{Team Coordination}
  \KwIn{the MMDP model and the current state $s$.}
  \KwOut{the best joint action $\vec{a}$.}
  \tcp{The task planning}
  $\{ t_i \} \gets$ compute the best task for each responder $i\in I$ \;
  \ForEach{$i\in I$} {
    \tcp{The path planning}
    $a_i \gets$ compute the best path to task $t_i$ \;
  }
  \Return{$\vec{a}$}
  \label{alg:coordination}
\end{algorithm}

\subsection{Task planning}
\label{sec:taskplanning}

As mentioned above, each player in the game owns a type of skill
and each task requires players with a certain combination of the
skills. The goal is to assign a task to each player that maximize
the overall performance given the current state. To do that, we
first compute all possible coalitions $\{ C_{jk} \}$ for each task
$j$ where a coalition $C_{jk} \subseteq I$ is a subset of the
players with the required skills. Then, we solve the following
optimization problem to find the best coalitions:
\begin{equation}
  \begin{array}{lll}
    \max_{x_{jk}} & \sum_{j, k} x_{jk} \cdot v(C_{jk}) & \\
    \mbox{s.t.} & x_{jk} \in \{0, 1\} & \\
    & \forall j, \sum_{k} x_{jk} \leq 1 & \mbox{(i)} \\
    & \forall i, \sum_{j, k} \delta_i(C_{jk}) \leq 1 & \mbox{(ii)}
  \end{array}
  \label{eq:cf}
\end{equation}
where $x_{jk}$ is the boolean variable to decide whether to select
coalition $C_{jk}$ for task $j$ or not, $v(C_{jk})$ is the
characteristic function for coalition $C_{jk}$, and
$\delta_i(C_{jk}) = 1$ if $i\in C_{jk}$ and 0 otherwise. Constraint
(i) ensures that a task $j$ is allocated at most to only one
coalition (a task does not need more than one group of players).
Constraint (ii) ensures that a player $i$ is assign to only one
task (a player cannot do more than one task at the same time). In
the optimization, we only consider the tasks that have not been
done and the players that are still alive. Because the players do
not change their skills and the requirements of the tasks are
static during the game, the set of all possible coalitions for each
task can be computed offline before the game.

The key challenge of the optimisation problem in
Equation~\ref{eq:cf} is to compute the value of the characteristic
function $v(C_{jk})$ for each coalition $C_{jk}$. This is the
expected value when the players in $C_{jk}$ are assigned to task
$j$. In order to compute this value, we need to know the plan after
the completion of task $j$ because not all tasks can be completed
in one shot. As aforementioned, computing the optimal plan is
intractable given the huge state space. Thus, we estimate the value
by simulations. The main process is outlined in
Algorithm~\ref{alg:tp}.

In each simulation, we first assign the responders in $C_{jk}$ for
task $k$ and run the simulator for this assignment starting from
the current state $s$. After task $k$ is completed, we collect the
reward $r$ and the new state $s'$. Then, we estimate the expected
value of $s'$ by Monte-Carlo Tree Search (MCTS), which has been
shown to be efficient for many large problems~\cite{?}. The basic
idea of MCTS is to maintain a search tree where each node is
associated with a state and each branch is an assignment for the
responders.

In the task planning level, ``completing a task by a responder'' is
treated as a macro action, assuming that each responder can find
his or her best way to the task and back to the closest dropoff
zone (computed by Section~\ref{sec:pathplanning}). Thus, the main
step of MCTS is to compute an assignment for the free responders at
each node of the search tree. We use Equation~\ref{eq:cf} to
compute the assignment and use the UCT heuristic~\cite{?} to
estimate the coalition value:
\begin{equation}
  v(C_{jk}) = \overline{v(C_{jk})} + c\sqrt{\frac{2N(s)}{N(s, C_{jk})}}
\end{equation}
where $\overline{v(C_{jk})}$ is the averaged value of coalition
$C_{jk}$ at state $s$, $c$ is a constant, $N(s)$ is the visiting
frequency of state $s$, and $N(s, C_{jk})$ is the frequency that
coalition $C_{jk}$ has been selected at state $s$. Intuitively, if
a coalition $C_{jk}$ has large averaged value or is rarely
selected, it gets high chance to be chosen in the next visit of the
node. Once a coalition $C_{jk}$ has been selected by the heuristic,
the responders in $C_{jk}$ will be assigned to the corresponding
task, i.e., task $k$. The constant $c$ is a tradeoff of the
exploitation and exploration for the UCT heuristic.

\begin{algorithm}[t]
  \caption{Task Planning}
  \KwIn{the current state $s$,
  a set of unfinished tasks $T$,
  and a set of free responders $I$.}
  \KwOut{a task assignment for all responders.}
  $\{ C_{jk} \} \gets$ compute all possible coalitions of $I$ for
  $T$ \;
  \ForEach{$C_{jk} \in \{C_{jk}\}$}{
    \tcp{The $N$ trial simulations}
    \For{$i=1$ \KwTo $N$}{
        $(r, s') \gets$ simulate the process with the starting \\\Indp state $s$
        until task $k$ is completed by the responders in $C_{jk}$ \;
        $V(s') \gets$ estimate the value of $s'$ with MCTS \Indm \;
        $v_i(C_{jk}) \gets r + \gamma V(s')$ \;
    }
    $v(C_{jk}) \gets \frac{1}{N} \sum_{i=1}^{N} v_i(C_{jk})$ \;
  }
  \Return the task assignment computed by Equation~\ref{eq:cf}
  \label{alg:tp}
\end{algorithm}

%[[More detail about sampling will be added.]]

\subsection{Path planning}
\label{sec:pathplanning}

In the path planning, we compute the best path for a player given
the location of his assigned task. This is a single-agent problem
that can be modeled as a MDP, $\langle S_i, A_i, P_i, R_i \rangle$,
where:
\begin{itemize}
  \item $S_i = S_r \times S_{p_i}$ is the state space. Player
      $i$ only need to consider his own state variable
      (location and health level) and the state variable of the
      radiation cloud.
  \item $A_i$ is the set of actions $i$. Player $i$ only need
      to consider the actions of staying in the same grid or
      moving to the 8 neighboring grids.
  \item $P_i = P_r \times P_{p_i}$ is the transition function.
      Player $i$ only need to consider the expanding of the
      radiation cloud and the change of his location and health
      level when moving in the grid map.
  \item $R_i$ is the reward function. Player $i$ has a small
      cost for moving around and a large penalty for being
      killed by entering the radiation cloud.
\end{itemize}
This process terminates when the location of the assigned task is
reached or the player is killed (the health level is 0) by the
radiation cloud. This is a typical MDP and can be solved by many
state-of-the-art MDP solvers~\cite{?}. The key challenge is that
the state space will blow up very quickly given a large map. To
address this, we adopt Real-Time Dynamic Programming
(RTDP)~\cite{?}. The main process is outline in
Algorithm~\ref{alg:pp}.

Instead of considering the whole state space, RTDP only visits the
states that are reachable from the starting state $s^0$. If the
loop cannot terminate in a fixed number of iterations, we assume
there is not valid path between the two locations (either there are
obstacle between them or the responder is killed by the radiative
cloud on the path). There are several techniques we used to speed
up the convergency of RTDP. In our problem, the map is static.
Thus, we can initialize $V(s)$ by the cost value of the map,
computed offline without considering the radiative cloud. This can
be done by Value Iteration~\cite{?} assuming that any location on
the map is the goal. The benefit of so doing is to help RTDP
quickly navigate among the obstacles (e.g., buildings, water pools,
blocked roads) without getting trapped in dead ends. Another
technique is that, when traversing the reachable states ($s'\in
S_i$ in Algorithm~\ref{alg:pp}), we only consider the responder's
current location and his or her 8 neighboring grids since
$P_i(s'|s,a) = 0$ for other locations. This will further speed up
the algorithm because, as mentioned above, the main challenge of
path planning in our problem is the huge state space.

\begin{algorithm}[t]
  \caption{Path Planning}
  \KwIn{the starting state $s^0$ and the goal state $s^g$.}
  \KwOut{a path from the starting location to the goal.}
  $s \gets s^0$ \;
  \Repeat{$s = s^g$}{
    \ForEach{$a\in A_i$}{
        $Q(s, a) \gets R_i(s, a) + \sum_{s'\in S_i} P_i(s'|s, a)
        V(s')$ \;
    }
    $a \gets \arg\max_{a'\in A_i} Q(s, a')$ \;
    $V(s) \gets Q(s, a)$ \;
    $s' \sim P_i(s'|s, a)$ \;
    $s \gets s'$ \;
  }
  \Return{$Q$}
  \label{alg:pp}
\end{algorithm}

%[[More detail about solving MDP will be added.]]
