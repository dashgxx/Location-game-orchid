% This is "aamas2014.tex", a revised version of aamas2013.tex
% This file should be compiled with "aamas2014.cls" 
% This example file demonstrates the use of the 'aamas2014.cls'
% LaTeX2e document class file. It is for those submitting
% articles to AAMAS 2014  conference. This file is based on
% the sig-alternate.tex example file.
% The 'sig-alternate.cls' file of ACM will produce a similar-looking,
% albeit, 'tighter' paper resulting in, invariably, fewer pages.
% than the original style ACM style.
%
% ----------------------------------------------------------------------------------------------------------------
% This .tex file (and associated .cls ) produces:
%       1) The Permission Statement
%       2) The Conference (location) Info information
%       3) The Copyright Line with AAMAS data
%       4) NO page numbers
%
% as against the acm_proc_article-sp.cls file which
% DOES NOT produce 1) through 3) above.
%
% Using 'aamas2014.cls' you don't have control
% from within the source .tex file, over both the CopyrightYear
% (defaulted to 200X) and the IFAAMAS Copyright Data
% (defaulted to X-XXXXX-XX-X/XX/XX).
% These information will be overwritten by fixed AAMAS 2014  information
% in the style files - it is NOT as you are used with ACM style files.
%
% ---------------------------------------------------------------------------------------------------------------
% This .tex source is an example which *does* use
% the .bib file (from which the .bbl file % is produced).
% REMEMBER HOWEVER: After having produced the .bbl file,
% and prior to final submission, you *NEED* to 'insert'
% your .bbl file into your source .tex file so as to provide
% ONE 'self-contained' source file.
%

% This is the document class for full camera ready papers and extended abstracts repsectively 

\documentclass{aamas2014}

% if you are using PDF LaTex and you cannot find a way for producing
% letter, the following explicit settings may help
 
\pdfpagewidth=8.5truein
\pdfpageheight=11truein

\begin{document}

% In the original styles from ACM, you would have needed to
% add meta-info here. This is not necessary for AAMAS 2014  as
% the complete copyright information is generated by the cls-files.


\title{AtomicOrchid: Mixed-Initiative Coordination for Disaster Response}

% AUTHORS


% For initial submission, do not give author names, but the
% tracking number, instead, as the review process is blind.

% You need the command \numberofauthors to handle the 'placement
% and alignment' of the authors beneath the title.
%
% For aesthetic reasons, we recommend 'three authors at a time'
% i.e. three 'name/affiliation blocks' be placed beneath the title.
%
% NOTE: You are NOT restricted in how many 'rows' of
% "name/affiliations" may appear. We just ask that you restrict
% the number of 'columns' to three.
%
% Because of the available 'opening page real-estate'
% we ask you to refrain from putting more than six authors
% (two rows with three columns) beneath the article title.
% More than six makes the first-page appear very cluttered indeed.
%
% Use the \alignauthor commands to handle the names
% and affiliations for an 'aesthetic maximum' of six authors.
% Add names, affiliations, addresses for
% the seventh etc. author(s) as the argument for the
% \additionalauthors command.
% These 'additional authors' will be output/set for you
% without further effort on your part as the last section in
% the body of your article BEFORE References or any Appendices.

%\numberofauthors{8} %  in this sample file, there are a *total*
% of EIGHT authors. SIX appear on the 'first-page' (for formatting
% reasons) and the remaining two appear in the \additionalauthors section.
%

\numberofauthors{1}

\author{
% You can go ahead and credit any number of authors here,
% e.g. one 'row of three' or two rows (consisting of one row of three
% and a second row of one, two or three).
%
% The command \alignauthor (no curly braces needed) should
% precede each author name, affiliation/snail-mail address and
% e-mail address. Additionally, tag each line of
% affiliation/address with \affaddr, and tag the
% e-mail address with \email.
% 1st. author
\alignauthor
Paper  XXX
%Ben Trovato\titlenote{Dr.~Trovato insisted his name be first.}\\
%       \affaddr{Institute for Clarity in Documentation}\\
%       \affaddr{1932 Wallamaloo Lane}\\
%       \affaddr{Wallamaloo, New Zealand}\\
%       \email{trovato@corporation.com}
% 2nd. author
%\alignauthor
%G.K.M. Tobin\titlenote{The secretary disavows any knowledge of this author's actions.}\\
%       \affaddr{Institute for Clarity in Documentation}\\
%       \affaddr{P.O. Box 1212}\\
%       \affaddr{Dublin, Ohio 43017-6221}\\
%       \email{webmaster@marysville-ohio.com}
% 3rd. author
%\alignauthor Lars Th{\o}rv{\"a}ld\titlenote{This author is the one who did all the really hard work.}\\
%       \affaddr{The Th{\o}rv{\"a}ld Group}\\
%       \affaddr{1 Th{\o}rv{\"a}ld Circle}\\
%       \affaddr{Hekla, Iceland}\\
%       \email{larst@affiliation.org}
}

%\and  % use '\and' if you need 'another row' of author names

% 4th. author
%\alignauthor Lawrence P. Leipuner\\
%       \affaddr{Brookhaven Laboratories}\\
%       \affaddr{Brookhaven National Lab}\\
%       \affaddr{P.O. Box 5000}\\
%       \email{lleipuner@researchlabs.org}

% 5th. author
%\alignauthor Sean Fogarty\\
%       \affaddr{NASA Ames Research Center}\\
%       \affaddr{Moffett Field}\\
%       \affaddr{California 94035}\\
%       \email{fogartys@amesres.org}

% 6th. author
%\alignauthor Charles Palmer\\
%       \affaddr{Palmer Research Laboratories}\\
%      \affaddr{8600 Datapoint Drive}\\
%       \affaddr{San Antonio, Texas 78229}\\
%       \email{cpalmer@prl.com}

%\and

%% 7th. author
%\alignauthor Lawrence P. Leipuner\\
%       \affaddr{Brookhaven Laboratories}\\
%       \affaddr{Brookhaven National Lab}\\
%       \affaddr{P.O. Box 5000}\\
%       \email{lleipuner@researchlabs.org}

%% 8th. author
%\alignauthor Sean Fogarty\\
%       \affaddr{NASA Ames Research Center}\\
%       \affaddr{Moffett Field}\\
%       \affaddr{California 94035}\\
%       \email{fogartys@amesres.org}

%% 9th. author
%\alignauthor Charles Palmer\\
%       \affaddr{Palmer Research Laboratories}\\
%       \affaddr{8600 Datapoint Drive}\\
%       \affaddr{San Antonio, Texas 78229}\\
%       \email{cpalmer@prl.com}

%}

%% There's nothing stopping you putting the seventh, eighth, etc.
%% author on the opening page (as the 'third row') but we ask,
%% for aesthetic reasons that you place these 'additional authors'
%% in the \additional authors block, viz.
%\additionalauthors{Additional authors: John Smith (The Th{\o}rv{\"a}ld Group,
%email: {\texttt{jsmith@affiliation.org}}) and Julius P.~Kumquat
%(The Kumquat Consortium, email: {\texttt{jpkumquat@consortium.net}}).}
%\date{30 July 1999}
%% Just remember to make sure that the TOTAL number of authors
%% is the number that will appear on the first page PLUS the
%% number that will appear in the \additionalauthors section.

\maketitle

\begin{abstract}
A paper on AtomicOrchid
\end{abstract}

% Note that the category section should be completed after reference to the ACM Computing Classification Scheme available at
% http://www.acm.org/about/class/1998/.

\category{H.4}{Information Systems Applications}{Multi-Agent Systems}

%A category including the fourth, optional field follows...
%\category{D.2.8}{Software Engineering}{Metrics}[complexity measures, performance measures]

%General terms should be selected from the following 16 terms: Algorithms, Management, Measurement, Documentation, Performance, Design, Economics, Reliability, Experimentation, Security, Human Factors, Standardization, Languages, Theory, Legal Aspects, Verification.

\terms{Design, Human Factors, Algorithms}

%Keywords are your own choice of terms you would like the paper to be indexed by.

\keywords{Human-Agent Interaction, Coordination, Decision under Uncertainty}

\section{Introduction}
Gopal
\begin{enumerate}
\item Introduce the challenge of teaming for disaster response
\item Complexity of decision making: planning + task allocation (e.g., CFST, fms, scerri's work)
\item Challenge of agent-based guidance.
\item Contributions: a platform for the study of mixed-initiative coordination, algorithm for coordination under uncertainty, evaluation of the algorithm (in simulation), evaluation of the platform with real users in 2 pilots - quantitative and qualitative results.
\end{enumerate}
\section{The Scenario}
Joel and Feng
\begin{enumerate}
\item Explain the disaster response problem.
\item Mathematical model of the problem
\end{enumerate}
\section{Team Coordination Algorithm}
Feng and Gopal
\begin{enumerate}
\item Feng's algorithm
\item experimental results in simulation - computational performance + no. of tasks completed in simulated settings. If possible, compare against something else.
\end{enumerate}
\section{The Atomic Orchid Platform}
Joel and Wenchao
\begin{enumerate}
\item explain the main components  and how agent is integrated
\item explain the instructions given to participants and how it mimics the disaster response problem detailed above.
\end{enumerate}
\section{Pilot Study}
Joel and Wenchao
\begin{enumerate}
\item Explain setup of experiment - area of interest + setup of tasks
\item Explain evaluation = quantitative and qualitative.
\end{enumerate}
\subsection{Conclusions}
\bibliography{citations}
\end{document}
