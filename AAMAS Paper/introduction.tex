\section{Introduction}
The coordination of teams of field responders in search and rescue missions is regarded as one  of the grand challenges for multi-agent systems research \cite{kitano:2001}. In such settings, responders with different capabilities (e.g., fire extinguishing, digging, or life support) have to form teams in order to perform rescue tasks (e.g., extinguishing a fire or digging civilians out of rubble or both) to minimise  costs (e.g., time or money) and maximise the number of lives and buildings saved. Thus, responders have to plan their paths to the tasks (as these may be distributed in space) and form specific teams  to complete some tasks. These teams, in turn, may  need to disband and reform other teams to complete other tasks requiring different capabilities, taking into account the status  of the tasks  (e.g., health of victims or building fire) and the environment (e.g., if a fire or radioactive cloud is spreading). Furthermore, uncertainty in the environment (e.g., wind direction or speed) or in the responders' abilities to complete tasks (e.g., some may be tired or get hurt) means that plans may need to change dynamically depending on the state of the players and the environment. 

To address these challenges, in recent years, a number of algorithms and mechanisms have been developed to create teams and allocate tasks. For example, \cite{ramchurn:etal:2010,Scerri2005} and \cite{ramchurn:etal:2010b,Chapman2009}, developed centralised and decentralised optimisation algorithms respectively to allocate rescue tasks efficiently to teams of field responders with different capabilities. However, none of these approaches considered the uncertainty in the environment or in the field responders' abilities. Crucially, to date, while all of these algorithms have been shown to perform well in simulations (representing responders as computational entities), none of them have been \emph{deployed} to guide \emph{real} human responders (amateur or expert) in real-time rescue missions. Thus, it is still unclear whether these algorithms will cope with real-world uncertainties (e.g., communication breakdowns or change in wind direction), be acceptable to humans (i.e., agent-computed plans are not confusing and take into account human capabilities), and do help humans perform better than on their own.

Against this background, in this paper we develop a novel algorithm for team coordination under uncertainty and evaluate it within a real-world mixed-reality game that embodies the simulation of team coordination in disaster response settings. In more detail, we consider scenario involving rescue tasks distributed in a disaster space over which a radioactive cloud is spreading. Tasks need to be completed by the responders before the area is completely covered by the cloud (as responders will die from radiation exposure) which is spreading according to varying wind speed and direction. Our algorithm captures the uncertainty in the scenario (i.e., in terms of environment and player states) and  is able to compute a policy to allocate responders to tasks to minimise the time to complete all tasks without them being exposed to significant radiation. The algorithm is then used by an agent to guide human responders based on their perceived states. This agent is then implemented in our deployed platform, AtomicOrchid, that structures the interaction between human responders, a human commander, and a planning agent in a mixed-reality location-based game. By so doing, we are able to study, both quantitatively and qualitatively, the performance of a mixed-initiative team (i.e., a human team under human and agent guidance)  and the interactions between the different actors in the system. Thus, this paper advances the state of the art in the following ways:
\begin{enumerate}
\item We develop a novel approximate algorithm for team formation under uncertainty using a Multi-agent Markov Decision Process (MMDP) paradigm, and show how it accounts for real-world uncertainties.
\item We present AtomicOrchid, a novel platform to evaluate team formation under uncertainty using the concept of mixed-reality games. AtomicOrchid allows an agent, using our team formation algorithm, to coordinate, in real-time, human players using mobile phone-based messaging, to complete rescue tasks efficiently.
\item We provide the first real-world evaluation of a team formation agent in a disaster response setting in multiple field trials and present both quantitative and qualitative results. Our results allow us to elucidate some of the challenges for the formation of human-agent collectives, that is, mixed-initiative teams where control can be passed between agents and humans in flexible ways.
\end{enumerate}
When taken together, our results show, for the first time, how agent-based coordination algorithms for disaster response can be validated in the real-world. Moreover, these results allow us to derive a methodology and guidelines to evaluate human-agent interaction in real-world settings. 

The rest of this paper is structured as follows. Section \ref{sec:scenario} formalises the disaster response problem as an MMDP. Section \ref{sec:algo} then describes the algorithm to solve the path planning and task allocation problems presented by the MMDP while Section \ref{sec:atomicorchid} describes the AtomicOrchid platform. Section \ref{sec:evaluation} presents our pilot study and the evaluation of the system in a number of field trials.  Finally, Section \ref{sec:conclusions} concludes.