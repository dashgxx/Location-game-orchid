\subsection{Radiation Cloud Modelling}
The radiation cloud diffusion process is modelled as a non-linear Markovian field stochastic differential 
equation (which assumes the cloud is Gaussian distributed in log-space).  The cloud is driven by wind 
forces which vary both spatially and temporally.  The wind velocity is modelled by two a priori independent 
Gaussian processes, one GP for each Cartesian coordinate axis.  The GP captures both the spatial distribution 
of wind velocity and also the dynamic process resulting from shifting wind patterns such as short term gusts 
and longer term variations.  Both the radiation cloud and wind model priors are combined into a single joint 
model called a Latent Force Model (Alvarez09) and predictions of the radiation cloud intensity are inferred 
using the Kalman filter.  The KF provides both an estimate of the radiation cloud and wind conditions as well 
as an indication of uncertainty in these estimates.  The KF state represents both the wind velocity and the log 
radiation cloud density on a regular grid defined across the environment.  The temporal component of the wind 
GP models are assumed Markovian and are thus converted to KF process models and corresponding process noise
covariance matrices.   The corresponding wind KF prediction models, defined for all points on the grid, are 
specified as per the KFGP (Reece10) and assume identical prior spatial GP models at consecutive time instances 
but with a multiplicative correlation factor between them to model the non-stationary wind dynamics.  

