\section{Discussion and Conclusions}\label{sec:conclusions}
\noindent In this paper we developed a novel approach to evaluating human-agent collaboration in uncertain, dynamic, and real-time environments. We developed a novel planning agent (using an MMDP approach) that is able to allocate tasks to teams in real-time and showed how it outperformed other baseline approaches. Furthermore we demonstrated how such a planning agent could be used by a human commander in AtomicOrchid, a mixed-reality game that simulates a real-world disaster and requires participants to collaborate and perform tasks in a physical environment. 

Results from our field trials indicate  the planning agent instructed players to carry out successful plans (outperforming a no-agent setting in terms of tasks completed and responders unharmed). The agent's ability to re-plan  as per responders' preferences and constraints was particularly effective as evidenced by the acceptance levels of agent-based instructions. In particular, based on our analysis, we propose the following design guidelines for human-agent collaboration in Human-Agent Collectives:\\

\noindent \textbf{Adaptivity}: our experience with AtomicOrchid suggests that planning algorithms should be designed to take in human input, and more importantly, be \emph{responsive} to the needs of the users. As we saw in AtomicOrchid, players repeatedly requested new tasks and this would not have been possible unless our algorithm  was computationally efficient but could dynamically assimilate updates, requests, and constraints. We believe this makes the algorithm more acceptable to the users. However, this adaptivity does cost the system in terms of efficiency as the rejection of tasks may lead the problem to be so constrained that the algorithm cannot return any solutions. To alleviate such issues, we believe human mediation may be important in nudging the players to justify their rejection of tasks or to nudge them not to do so too frequently. \\

\noindent \textbf{Interaction Simplicity}: our agent was designed to issue simple commands (Do X with Y) and respond to simple requests (OK or Reject Task). Such simple messages were shown to be far more effective at guiding players to do the right task than the unstructured human communication in the non-agent assisted case that was fraught with inconsistencies and inaccuracies. In fact, we would suggest that agents should be designed with minimal options to simplify the reasoning users have to do to interact with the agent, particularly when they are under pressure to act. However, interaction simplicity in this context, to us also means providing human responders with interactive abilities to do what they are good at: dealing with unforeseen contingencies. Hence, it is important to provide unconstrained communication means such as chat, walkie talkies or mobile phones in addition to the `simple' instructions that the agent provides. In effect, we are designing an interactional setting in which the agent is dealing with the routine and predictable aspects of the setting (repetitive tasks assignments), and the human coordinators in the HQ are freed up to deal with contingencies and the less predictable complexities as and when they arise.\\

\noindent \textbf{Flexbile autonomy}: the HQ dashboard proved to be a key tool for the HQ coordinator $H$ to \emph{check} and \emph{correct for} the allocations of $PA$, taking into account the real-world constraints that the players on the ground faced. In particular, letting the human oversee the agent (i.e., ``on-the-loop") at times and actively instructing  the players (and bypassing the agent) at other times (i.e., ``in-the-loop") as and when needed, was seen to be particularly effective. This was achieved by $H$ \emph{without the agent} defining when such transfers of control should happen (as in \cite{scerri:etal:2005}) and, therefore, left the coordinator the option of taking control when she judged it was needed. However, while this allows humans to \emph{choose} what to do, it is not clear whether they would have been better off going with the agent's plan. Hence, we suggest that such deployed autonomous systems should be built for flexible autonomy. Specifically, interfaces should be designed to pass control \emph{seamlessly} between humans and agents and the implications of human-based ``corrective'' actions  should be made explicit to the humans to ensure they know when to take control, and when to let the agent decide.

Much remains to be done to further validate agent-based planning in real-world disaster response given that field trials of AtomicOrchid are limited to using volunteers and in settings that only approximate the typical environments faced by emergency responders. Hence, in future work, building upon our ethnographic studies of real-world training exercises \cite{fischer:etal:2015}, we aim to deploy our planning agent to support expert emergency responders from RescueGlobal during their annual multi-national disaster response training exercise  (Angel Thunder\footnote{http://www.dm.af.mil/library/angelthunder2013.asp.}). By so doing, we will develop new requirements for agent-based technologies for settings where users are highly trained and the actions taken by the agent can have major impact on the performance of the team (e.g., leading to loss of lives or waste of real resources).
\section*{Acknowledgements}
This work was done as part of the EPSRC-funded ORCHID project (EP/I011587/1). We also wish to thank Trung Dong Huynh for generating the traces of the player movements as well as Davide Zilli and Sebastian Stein for initial input on the Responder App. Finally we wish thank the anonymous reviewers for their constructive comments that helped improve the paper.