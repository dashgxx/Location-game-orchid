\section{Conclusions}\label{sec:conclusions}
\noindent In this paper we developed a novel approach for integrating and evaluating agent-based coordination algorithms that allocate teams of emergency responders in dynamic and uncertain environments.  In particular, we conducted field-trials of a task planning agent using a mixed-reality game  called AtomicOrchid in order to focus on the issues that arise in human-agent collaboration in team coordination. Results from our study indicate  the planning agent instructed players to carry out successful plans (outperforming a no-agent setting in terms of tasks completed and responders unharmed). The agent's ability to re-plan  as per responders' preferences and constraints was particularly effective. Finally our results  suggest that systems involving human-agent collaboration should be adaptive, involve simple  interactions between humans and agents, and allow for flexible autonomy. Future work will look at running AtomicOrchid with expert responders and  exploring different interactional arrangements of humans and agents, in particular where control may be distributed across the team.\vspace{-1mm}