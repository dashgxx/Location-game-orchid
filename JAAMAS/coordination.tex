\section{Team Coordination Algorithm}\label{sec:algo}
\noindent Unfortunately, as in most MDP-based approaches to solving team coordination problems (see Section \ref{sec:decisiontheoretic}, our  MMDP model results in a very large search space,
even for small-sized problems. For example, with 8 responders and
17 tasks in a 50$\times$55 grid, the number of possible states is
more than $2\times 10^{400}$. Therefore, it is practically
impossible to compute the optimal solution. In such cases, we need
to consider approximate solutions that result in high quality
allocations.  To this end, we develop an
approximate solution using  the observation that responders first need to {\em cooperatively}  form teams (i.e., agree on who will do what),
and  that they can then {\em independently} compute the best path to the task.
In our planning algorithm, we use this observation to decompose the
decision-making process into a hierarchical structure with two
levels: at the top level, a task planning algorithm is run for the
whole team to assign the best task to each responder given the
current state of the world; at the lower level, given a task, a
path planning algorithm is run by each responder to find the best
path to the task from her current location.

Furthermore, not all states of MMDPs are relevant to the problem
(e.g., if a responder gets injured, she is incapable of doing any
task in the future and therefore her state is irrelevant
to other responders) and we only need to consider the reachable
states given the current global state $S$ of the problem. Hence,
given the current state, we compute the policy online only for
reachable states. This saves a considerable amount of computation
because the size of the reachable states is usually much smaller
than the overall state space. For example, given the current
location of a responder, the one-step reachable locations are the 8
neighbouring locations plus the current locations, which are 9
locations out of the 50$\times$55 grid. Jointly, the reduction is
huge, from $(50\times 55)^8$ to $9^8$ for 8 responders. Another
advantage of online planning is that it allows us to refine the
model as more information is obtained or unexpected events happen.
For example, given that the wind speed or  direction 
 may change, the uncertainty about the radioactive cloud may increase.
If a responder becomes tired, the outcome of  her actions may
be liable to greater uncertainty.

The main process of our online hierarchical planning algorithm is
outlined in Algorithm~\ref{alg:coordination}. The following
sections describe the procedures of each level in more detail.

\begin{algorithm}[t]
  \caption{Team Coordination Algorithm}
  \label{alg:coordination}
  \Indm
  \KwIn{the MMDP model and the current state $s$.}
  \KwOut{the best joint action $\vec{a}$.}
  \Indp\BlankLine
  \tcp{The task planning}
  $\{ t^i \} \gets$ compute the best task for each responder $p_i\in I$ \;
  \ForEach{$p_i\in I$} {
    \tcp{The path planning}
    $a_i \gets$ compute the best path to task $t^i$ \;
  }
  \Return{$\vec{a}$}
\end{algorithm}


\subsection{Task Planning}
\label{sec:taskplanning}
\noindent As described in Section \ref{sec:model}, each responder
$p_i$ is of a specific type $\theta_i \in \Theta$ that determines which task
she can perform and  a task $t$ can only be completed by a team of
responders with the required types $\Theta_t$. If, at some point in
the execution of a plan, a responder $p_i$ is incapable of
performing a task (e.g., because she is tired or suffered a high
radiation dose), she will be removed from the set of responders
under consideration (that is $I \to I \setminus p_i$). This
information can be obtained from the state $s \in S$. When a task
is completed by a chosen team, the task is simply removed from the
set (that is $T \to T\setminus t_k$ if $t_k$ has been completed).

Now, to capture the efficiency of groupings of responders at
performing tasks, we define the value
of a team $v(C_{jk})$ that reflects the level of performance of
team $C_k$ in performing task $t_j$. This is computed from the estimated rewards the team obtains for performing $t_j$ (as we show below).  Then, the goal of the task
planning algorithm is to assign a task to each team that maximises
the overall team performance given the current state $s$, i.e.,
$\sum_{j=1}^m v(C_{j})$ where $C_j$ is a team for task $t_j$ and $\{
C_1, \cdots, C_m \}$ is a {\em partition} of $I$ ($\forall j\neq
j', C_j \bigcap C_{j'} = \emptyset$ and $\bigcup_{j=1}^m C_j=I$).
In what follows, we first detail the procedure to compute the value
of all teams that are valid in a given state and then proceed to
detail the main algorithm to allocate tasks. Note that these
algorithms take into account the uncertainties captured by the
transition function of the MMDP.


\subsubsection{Team Value Calculation}
\noindent The computation of  $v(C_{jk})$ for each team
$C_{jk}$ is challenging because not all tasks can be completed by
one allocation (there are usually more targets than responders). Moreover, the policy after completing task $t_j$ must also be computed by the agent, which is time-consuming given the number of states and joint
actions. Given this, we propose to estimate $v(C_{jk})$ through
several simulations. This is much cheaper computationally as it avoids computing the complete policy to come up
with a good estimate of the team value, though we may not be able to
evaluate all possible future outcomes. According to the central
limit theorem, if the number of simulations is sufficiently
large, the estimated value will converge to the true $v(C_{jk})$.
This process is outlined in Algorithm~\ref{alg:taskplanning}.
\begin{algorithm}[htbp]
  \caption{Team Value Calculation}
  \label{alg:taskplanning}
  \Indm
  \KwIn{the current state $s$,
  a set of unfinished tasks $T$,
  and a set of free responders $I$.}
  \KwOut{a task assignment for all responders.}
  \Indp\BlankLine
  $\{ C_{jk} \} \gets$ compute all possible teams of $I$ for
  $T$ \;
  \ForEach{$C_{jk} \in \{C_{jk}\}$}{
    \tcp{The $N$ trial simulations}
    \For{$i=1$ \KwTo $N$}{
        $(r, s') \gets$ simulate the process with the starting state $s$
        until task $k$ is completed by the responders in $C_{jk}$ \;
        \If{$s'$ is a terminal state} {
            $v_i(C_{jk}) \gets r$ \;
        } \Else {
            $V(s') \gets$ estimate the value of $s'$ with MCTS \;
            $v_i(C_{jk}) \gets r + \gamma V(s')$ \;
        }
    }
    $v(C_{jk}) \gets \frac{1}{N} \sum_{i=1}^{N} v_i(C_{jk})$ \;
  }
  \Return the task assignment computed by Equation~\ref{eq:cf}
\end{algorithm}

In each simulation of Algorithm~\ref{alg:taskplanning}, we first
assign the responders in $C_{jk}$ to task $t_j$ and run the
simulator starting from the current state $s$ (Line 4). After task
$t_j$ is completed, the simulator returns the sum of the rewards
$r$ and the new state $s'$ (Line 4). If all the responders in
$C_{jk}$ are incapable of doing other tasks (e.g., suffered
radiation burns), the simulation is terminated (Line
5). Otherwise, we estimate the expected value of $s'$ using
Monte-Carlo Tree Search (MCTS)~\cite{kocsis2006bandit} (Line 8),
which provides a good trade-off between exploitation and exploration
of the policy space and has been shown to be efficient for large
MDPs.\footnote{Other methods such as sequential greedy assignment
or swap-based hill climbing~\cite{proper2009solving} may also be useful. However, they do not explore the policy space as well as MCTS \cite{kocsis2006bandit}.} After $N$ simulations, the average value is returned as an approximation of
the team value (Line 10).

The basic idea of MCTS is to maintain a search tree where each node
is associated with a state $s$ and each branch is a task assignment
for all responders. To implement MCTS, the main step is to compute
an assignment for the free responders (a responder is free when she
is capable of doing tasks but not assigned to any) at each
node of the search tree. This can be computed by
Equation~\ref{eq:cf} using the team values estimated by the UCB1
heuristic~\cite{auer2002finite} to balance exploitation and
exploration:
\begin{equation}
  v(C_{jk}) = \overline{v(C_{jk})} + c\sqrt{\frac{2N(s)}{N(s, C_{jk})}}
\end{equation}
where $\overline{v(C_{jk})}$ is the averaged value of team $C_{jk}$
at state $s$ so far, $c$ is a trade-off constant, $N(s)$ is the
visiting frequency of state $s$, and $N(s, C_{jk})$ is the
frequency that team $C_{jk}$ has been selected at state $s$.
Intuitively, if a team $C_{jk}$ has  a higher average value in the
trials so far or is rarely selected in the previous visits, it has
higher chance of being selected in the next visit of the tree node.

As we assume that the type of a responder and the role requirements
of each task are static, we can compute all possible team values
offline. Therefore, in the online phase, we only need to filter out
the teams for completed tasks and those containing
incapacitated responders to compute the team set $\{ C_{jk} \}$.

\subsubsection{Coordinated Task Allocation}
\noindent Given the team values computed above, we then solve the
following optimisation problem to find the best solution:
\begin{equation}
  \begin{array}{lll}
    \max\limits_{x_{jk}} & \sum_{j, k} x_{jk} \cdot v(C_{jk}) & \\[2pt]
    \mbox{s.t.} & x_{jk} \in \{0, 1\} & \\[2pt]
    & \forall j, \sum_{k} x_{jk} \leq 1 & \mbox{(i)} \\[2pt]
    & \forall i, \sum_{j, k} \delta_i(C_{jk}) \leq 1 & \mbox{(ii)}
  \end{array}
  \label{eq:cf}
\end{equation}
where $x_{jk}$ is the boolean variable to indicate whether team
$C_{jk}$ is selected for task $t_j$ or not, $v(C_{jk})$ is the value of team $C_{jk}$, and $\delta_i(C_{jk}) =
1$ if responder $p_i\in C_{jk}$ and 0 otherwise. In the
optimisation, constraint (i) ensures that a task $t_j$ is allocated
to at most one team (a task does not need more than one group of
responders) and constraint (ii) ensures that a responder $p_i$ is
assigned to only one task (a responder cannot do more than one task
at the same time). This is a standard Mixed Integer Linear Program
(MILP) that can be efficiently solved  using off-the-shelf solvers (e.g., IBM CPLEX or lp\_solve).

\subsubsection{Adapting to Responder Requests}\label{sec:adaptive}
\noindent An important characteristic of our approach is that it can easily
incorporate the preferences of the responders. For example, if a
responder declines a task allocated to it by the planning agent, we
simply filter out the teams for the task that contain this
responder. By so doing, the responder will not be assigned to the
task. Moreover, if a responder prefers to do the tasks with another
responder, we can increase the weights of the teams that
contain them in Equation~\ref{eq:cf} (by default, all teams
have identical weights of 1.0). Thus, our approach is adaptive to the
 preferences of human responders.

\subsection{Path Planning}
\label{sec:pathplanning}

\noindent In the path planning phase, we compute the best path for
a responder to her assigned task. This phase is stochastic as there
are uncertainties in the radioactive cloud and the responders'
actions. We model this problem as a single-agent MDP that can be
defined as a tuple, $\mathcal{M}_i = \langle S_i, A_i, P_i, R_i
\rangle$, where: (1) $S_i = S^G_r \times S_{p_i}$ is the state
space, (2) $A_i$ is the set of $p_i$'s actions, (3) $P_i = P_r
\times P_{p_i}$ is the transition function, and (4) $R_i$ is the
reward function. In this level, responder $p_i$ only needs to
consider the states of the radioactive cloud $S^G_r$ and her own
states $S_{p_i}$ and her moving actions. Similarly, the transition
function only needs to consider the spreading of the radioactive
cloud $P_r$ and the changes of her locations and health levels when
moving in the field $P_{p_i}$, and the reward function only needs
to consider the cost of moving to a task and the penalty of
receiving high radiation doses. This is a typical MDP that can be
solved by many existing solvers (see the most recent
survey~\cite{kolobov2012planning}). We choose Real-Time Dynamic
Programming (RTDP)~\cite{barto1995learning} because it is simple
and particularly fits our problem, that is, a goal-directed MDP
with large number of states. However, other approaches for solving
large MDPs  could equally be used here.

There are several techniques we use to speed up the convergence of
RTDP. In our problem, the map is static. Thus, we can initialize
the value function $V(s)$ using the cost of the shortest path
between the current location and the task location on the map, which
can be computed offline without considering the radioactive cloud.
This helps RTDP quickly navigate among the obstacles (e.g.,
buildings, water pools, blocked roads) without getting trapped in
dead-ends during the search. 

Since, in this paper, we focus on the integration and validation of the algorithm in a real-world deployment, we leave the presentation of  computational simulation results and comparisons with other agent-based planning solutions (using our MMDP formulation) to Appendix \ref{sec:appendix2}. As argued earlier (see Section \ref{sec:challenges}), while computational simulations are useful to exemplify extreme cases, they do not explain how human FRs and the planning agent will actually collaborate. Hence, we turn to the real-world trial of our algorithm as part of a planning agent next.

%Another speed up is also possible if, when traversing the possible states, we only consider the responder's current location and the neighbouring points. This will further speed up the algorithm where the main bottleneck is the huge state space.

