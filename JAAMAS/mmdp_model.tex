\subsection{The Optimisation Problem} \noindent Previous agent-based models for team
coordination in disaster response typically assume deterministic
task executions and environments
\cite{ramchurn:etal:2010,Scerri2005}. However, in order to evaluate
agent-guided coordination in a real-world environment, it is
important to consider uncertainties due to player behaviours and
the environment (as discussed in the previous section). Given this,
we propose a new representation for the task allocation problem in
disaster response that does take into account such uncertainties.
More specifically, we represent this problem using an MMDP that
captures the uncertainties of the radioactive cloud and the
responders' behaviours. We model the spreading of the radioactive
cloud as a random process over the disaster space and allow the
actions requested from the responders to  fail (because they
decline to go to a  task) or incur delays (because they are too
slow) during the rescue process. Thus in the MMDP model, we
represent  task executions as stochastic processes of state
transitions, while the uncertainties of the radioactive cloud and
the responders' behaviours can be easily captured with transition
probabilities.  More formally, the MMDP is represented by tuple
$\mathcal{M} = \langle I, S, \{A_i\}, P, R \rangle$, where $I = \{
p_1, p_2, \cdots, p_n \}$ is the set of actors as defined in the
previous section, $S$ is the state space, $A_i$ is a set of
responder $p_i$'s actions, $P$ is the transition function, and $R$
is the reward function. In the MMDP, the state is Markovian and
represents all the information that is sufficient for the agent to
make decision. The transition function models the dynamics of the
system and how it reacts to the responders' actions. The reward
function specifies the objective of the system (e.g., saving as
many lives as possible or minimising the radiation dose received). We elaborate on each of these below.

In more detail, $S= S^G_r \times S_{p_1} \times \cdots \times
S_{p_n} \times S_{t_1} \times \cdots \times S_{t_m}$ where:
\begin{itemize}
  \item $S^G_r = \{l_{(x,y)}| (x, y) \in G\}$ is the state
      variable of the radioactive cloud that specifies the
      radioactive level $l_{(x,y)}\in[0, 100]$ at every point
      $(x, y)\in G$.
  \item $S_{p_i} = \langle h_i, (x_i, y_i), t_j \rangle$ is the
      state variable for each responder $p_i$ that specifies
      her health level $h_i\in[0, 100]$, her present position
      $(x_i, y_i)$, and the task $t_j$ she is carrying out,
      which is ${\tt null}$ when she has no task.
  \item $S_{t_j} = \langle {\tt st_j}, (x_j, y_j) \rangle$ is
      then the state variable for task $t_j$ to specify its
      status ${\tt st_j}$ (i.e., the target is picked up,
      dropped off, or idle) and position $(x_j, y_j)$.
\end{itemize}

The three types of actions  (in set $A_i$) a responder can take
are: (i) {\em stay} in the current location $(x_i, y_i)$, (ii) {\em
move} to the 8 neighbouring locations, or (iii) {\em complete} a
task located at $(x_i, y_i)$. A joint action $\vec{a}=\langle a_1,
\cdots, a_n \rangle$ is a set of actions where $a_i\in A_i$, one
for each responder (a responder may just \emph{stay} at its current
position if it has no targets to rescue).

The transition function $P$ is defined in more detail as: $P= P_r
\times P_{p_1} \times \cdots \times P_{p_n} \times P_{t_1} \times
\cdots \times P_{t_m}$ where:
\begin{itemize}
    \item $P_r(s'_r|s_r)$ is the probability the radioactive
        cloud spreads from state $s_r\in S^G_r$ to $s'_r\in
        S^G_r$. It captures the uncertainty of the  radiation
        levels in the environment due to  noisy sensor readings
        and the variations in wind speed and direction.
    \item $P_{p_i}(s'_{p_i}|s, a_i)$ is the probability
        responder $p_i$ transitions to a new state $s'_{p_i}\in
        S_{p_i}$ when executing action $a_i$. For example, when
        a responder is asked to go to a new location,  she
        may not end up there because  she is tired,
        gets injured, or receives radiation doses that are life
        threatening.
    \item $P_{t_j}(s'_{t_j}|s, \vec{a})$ is the probability
        of task $t_j$ being completed. A task $t_j$ can only be completed by a
        team of responders with the required types ($\Theta_{t_j}$) located at the
        same position as $t_j$.
\end{itemize}

Now,  if task $t_j$ is completed (i.e., in ${\tt st_j}\in S_{t_j}$,
the status ${\tt st_j}$ is marked as ``dropped off'' and its
position $(x_j, y_j)$ is within a safe zone), the team will be
rewarded using function $R$. The team is penalised if a responder
$p_i$ gets injured or receives a high dose of radiation (i.e., in
$s_{p_i}$, the health level $h_i$ is 0). Moreover, we attribute a
cost to each of the responders' actions since  each  action
requires them to exert some effort (e.g., running or carrying
objects).

Give the above definitions, a policy for the MMDP is a mapping from
states to joint actions, $\pi: S \rightarrow \vec{A}$ so that the
responders know which actions to take given the current state of
the problem. The quality of a policy $\pi$ is  measured by
its expected value $V^\pi$, which can be computed recursively by
the Bellman equation:
\begin{equation}
  V^\pi(s^\tau) = R(s^\tau, \pi(s^\tau)) + \sum_{s^{\tau+1}\in S}
  P(s^{\tau+1}|s^\tau, \pi(s^t)) V^\pi(s^{\tau+1})
\end{equation}
where $\tau$ denotes the current time point and $\pi(s^\tau)$ is a
joint action given $s^\tau$. The goal of solving the MMDP is to
find an optimal policy $\pi^*$ that maximises the expected value
with the initial state $s^0$, $\pi^* = \arg\max_{\pi} V^\pi(s^0)$.

At each decision step, we assume that the PA can fully observe the
state of the environment $s$ by collecting sensor readings of the
radioactive cloud and GPS locations of the responders. Given a
policy $\pi$ of the MMDP, a joint action $\vec{a}=\pi(s)$ can be
selected and broadcast to the responders (as mentioned earlier).
