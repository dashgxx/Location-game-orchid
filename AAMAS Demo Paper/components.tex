

%The platform is built using the geoloqi platform, Sinatra for Ruby, and state-of-the-art web technologies such as socket.io, node.js and the Google Maps API. 
%
%\subsection{Integrating the Planning Agent}
%\noindent The planning agent $PA$ takes the game status (i.e., positions of players, known status of the cloud, and messages received from players) as input and produces a plan for each responder  for the current state. $PA$ is deployed on a separate server. The AtomicOrchid server requests a plan from the agent via a stateless HTTP interface by transmitting the game status in JSON format. Polling (and thus re-planning) is triggered by two types of game events:
%\begin{itemize}
%\item \textit{Completion of task}. On successful rescue of a target, a new plan (i.e., allocation of tasks to each responder) is requested from the agent.
%\item \textit{Explicit reject}. On rejection of a task allocation by any of the allocated first responders, a new plan is requested.  More importantly, the rejected allocation is, in turn, used as a constraint within the optimisation run by the planner agent (as described in Section \ref{sec:adaptive}). For example, if two responders, one medic and one soldier, were allocated a task and the medic rejected it, the planning agent would rerun the optimisation algorithm with the constraint that this medic should not be allocated this task. If another medic is free (and accepts) the soldier and this medic can go ahead and complete the task. Otherwise, a new plan is created for the soldier.
%\end{itemize} 
%
%%\subsection{Interacting with planning agent}
%%There can interact directly with field players through a task tab (Figure xx) and agent plans are also visible to HQ's dashboard interface.
%Once a plan is received from $PA$, the AtomicOrchid game engine splits the plan for a given team into individual task allocations for each player and sends them to their mobile responder app. The app presents the task allocation in the task tab, detailing: i) the responder to team up with, ii) the allocated target (using target id), and iii) approximate direction of the target (e.g., north, east).  Once a player accepts a task, an acknowledgement is sent to their teammate, while rejecting a task triggers a new assignment from $PA$. 

%Furthermore, $H$ is provided with a visualisation of task allocations for each player on demand (by button press), to help monitor the task allocation computed by the agent.

