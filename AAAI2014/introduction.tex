\section{Introduction}
\noindent The coordination of first responders in search and rescue missions is a grand challenge for multi-agent systems research \cite{kitano:2001}. In such settings, responders with different capabilities (e.g., fire-fighting or life support) have to form teams in order to perform rescue tasks  (e.g., extinguishing a fire or providing first aid) to minimise  loss of life and costs (e.g., time or money). These tasks may be geographically distributed  and require specific teams  to be completed. Furthermore, uncertainty in the environment (e.g., wind direction or spread of fire) or in the responders' abilities to complete tasks (e.g., some may be tired or get hurt) means that plans are likely to change continually to reflect the prevailing assessment of the situation. 

To address these challenges, a number of algorithms  have been developed to form teams and allocate tasks. For example, \cite{ramchurn:etal:2010,Scerri2005} and \cite{Chapman2009}, devised centralised and decentralised algorithms respectively to allocate rescue tasks to first responders with different capabilities. However, none of these approaches considered the inherent uncertainty in the environment or in the first responders' abilities. Crucially, to date, while all of these algorithms have been shown to perform well in simulations (representing responders as computational entities), none of them have been \emph{trialled} to guide \emph{real} human responders in real-time rescue missions. Thus, it is still unclear whether these algorithms will cope with real-world uncertainties (e.g., social preference), be acceptable to humans (i.e., be intelligible and effective), and actually augment, rather than hinder,  human performance.

Against this background, we develop a novel algorithm for team coordination under uncertainty and evaluate it within a real-world \emph{mixed-reality game} \cite{Fischer:etal:2012} that embodies the simulation of team coordination in disaster response settings. Specifically, our algorithm is used by a software agent to guide human responders in the game to stay clear of a virtual radioactive cloud and complete a number of geo-located rescue tasks in the real world. By so doing, we  study, both quantitatively and qualitatively, the performance of a human-agent collective (i.e., a mixed-initiative team where control can shift between humans and agents)  and the interactions between the different actors in the system. In particular, we  advance the state of the art in the following ways. First, we develop a novel representation for team coordination (i.e., path planning and task allocation) under uncertainty using Multi-agent Markov Decision Processes (MMDP)  \cite{boutilier1996planning}. Moreover, we provide an approximate algorithm to solve the MMDP and show how it  is adaptive to human requests to re-plan task allocations. Second, we present AtomicOrchid, a novel game to evaluate team coordination under uncertainty using the concept of mixed-reality games. AtomicOrchid allows a planning agent, using our task planning algorithm, to coordinate, in real-time, human players using mobile phone-based messaging, to complete rescue tasks efficiently. Third, we provide a real-world evaluation of our planning agent in a disaster response scenario in multiple field trials and present both quantitative and qualitative results. 
Our results show, for the first time, how agent-based coordination algorithms for disaster response can be integrated and validated with human teams. Moreover, these results allow us to derive  guidelines for systems involving  human-agent collaboration.  In what follows, we first formalise the disaster response problem as a MMDP and then describe the algorithms to solve the the MMDP. Given this we describe the AtomicOrchid platform and present results of our field trials and discuss our design guidelines for human-agent collaboration.\vspace{-2mm}


%%%%%%%%%%%%%%%%%%%%%%%%%%% OLD TEXT %%%%%%%%%%%%%%%%%%%%%%%%%%%%

%In more detail, we consider a scenario involving rescue tasks distributed in a physical space over which a (virtual) radioactive cloud is spreading. Tasks need to be completed by the responders before the area is completely covered by the cloud (as responders will die from radiation exposure) which is spreading according to varying wind speed and direction. Our algorithm captures the uncertainty in the scenario (i.e., in terms of environment and player states) and  is able to compute a policy to allocate responders to tasks that minimises task completion time and ensures responders are not exposed to significant radiation. The algorithm is then used by an agent to guide human responders. Specifically


%The rest of this paper is structured as follows. First we \ref{sec:scenario} formalises the disaster response problem as a MMDP. Section \ref{sec:algo}  describes the algorithms to solve the the MMDP while Section \ref{sec:atomicorchid}  details the AtomicOrchid platform. Section \ref{sec:evaluation} presents our field trials and Section \ref{sec:conclusions} concludes.