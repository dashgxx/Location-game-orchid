\subsubsection{The Optimisation Problem}
\label{sec:model}

Previous agent-based models for team coordination in disaster
response typically assume deterministic task executions and
environments \cite{ramchurn:etal:2010,Scerri2005}. However, in
order to evaluate agent-guided coordination in a real-world
environment, it is important to consider uncertainties due to
player behaviours and the environment. Given this, we represent the
task allocation problem using a {\em multi-agent Markov decision
process} (MMDP) that captures the uncertainties of the radioactive
cloud and the responders' behaviours. More specifically, we model
the spreading of the radioactive cloud as a random process over the
disaster space and allow the actions requested from the responders
to  fail (because they decline to go to a  task) or incur delays
(because they are too slow) during the rescue process. Thus in the
MMDP model, we represent  task executions as stochastic processes
of state transitions, while the uncertainties of the radioactive
cloud and the responders' behaviours can be easily captured with
transition probabilities.

More formally, the MMDP is represented by tuple $\mathcal{M} =
\langle I, S, \{A_i\}, P, R \rangle$, where $I$ is the set of
actors as defined in the previous section,  $S$ is the state space,
$A_i$ is a set of responder $p_i$'s actions, $P$ is the transition
function, and $R$ is the reward function. A policy $\pi$ is a
mapping from states to joint actions so that the responders know
which actions to take given the current state. The quality of a
policy $\pi$ is measured by its expected value $V^\pi$, which can
be computed recursively using the Bellman equation. The goal of
solving the MMDP is to find an optimal policy $\pi^*$ that
maximises the expected value with the initial state $s^0$. At each
decision step, we assume the planning agent can fully observe the
state of the environment $s$ by collecting sensor readings of the
radioactive cloud and GPS locations of the responders. Given a
policy $\pi$ of the MMDP, a joint action $\vec{a}$ can be selected
and broadcast to the responders.
